\newcommand{\letitle}{Vérification et preuve automatique d'appartenance d'un mot à une grammaire formelle.}
\newcommand{\leauthor}{Ulysse Durand}
\newcommand{\lesubject}{Preuve grammaire}

\documentclass[12pt]{beamer}
\usepackage{tikzit}

\usepackage[utf8]{inputenc}
\usepackage{amssymb}
\usepackage{amsmath}
\usepackage{xcolor}
\usepackage{bbold}
\usepackage[xcolor,leftbars]{changebar}

\setbeamertemplate{footline}[frame number]
\setcounter{secnumdepth}{1}

\definecolor{DarkBlue}{RGB}{0,16,80}
\newcommand{\norm}[1]{\lvert #1 \rvert}
\usepackage{array,multirow,makecell}
\usepackage{subfig}
\setcellgapes{1pt}
\makegapedcells
\newcolumntype{R}[1]{>{\raggedleft\arraybackslash }b{#1}}
\newcolumntype{L}[1]{>{\raggedright\arraybackslash }b{#1}}
\newcolumntype{C}[1]{>{\centering\arraybackslash }b{#1}}
\usepackage{hyperref}
\setlength{\parskip}{0.15cm}
\hypersetup{
    colorlinks=truem,
    linkcolor=black,
    filecolor=red,
    urlcolor=blue
}

\urlstyle{same}

\everymath{\displaystyle}
\title{\letitle}
\author{\leauthor}
\date{}
\begin{document}
\begin{frame}
\titlepage
\end{frame}
    \begin{frame}\frametitle{Les grammaires formelles}

$G = (T,N_t,S,D)$ où :
\begin{itemize}
\item $T$ est l'alphabet des terminaux
\item $N_t$ est l'alphabet des non terminaux
\item $S \in N_t$ est l'axiome

Notons $\Sigma := N_t \cup T$
\item $D \subset (\Sigma ^ \star )^2$ est l'ensemble des règles de dérivation.
\end{itemize}
\end{frame}

\begin{frame}\frametitle{Definitions}
\begin{itemize}
\item $x \overset{(a,b)}{\rightarrow} x' \iff x = uav \text{ et } x' = ubv$
\pause
\item $\rightarrow := \bigcup_{(a,b) \in D} \overset{(a,b)}{\rightarrow}$ 
\pause
\item $\overset{*}{\rightarrow}$ la cloture transitive et réflexive de $\rightarrow$ ($x \overset{*}{\rightarrow} m \iff x \rightarrow m_1 \rightarrow \dots \rightarrow m_{n-1} \rightarrow m$)
\pause
\item $\delta(x)$ successeurs de $x$ par $\rightarrow$ (ou successeurs directs de $x$ par $\overset{*}{\rightarrow}$)
\pause
\item $\norm{x}_l$ nombre d'occurences de la lettre $l$ dans x.
\pause
\item Le langage de la grammaire formelle  G est :
\begin{equation*}\mathcal{L} (G) := \delta(S) \cap T^\star\end{equation*}
\end{itemize}
\end{frame}

\begin{frame}\frametitle{Vérification de preuve}
$D = \{d_1,\dots,d_n\} = \{(a_1,b_1),\dots,(a_n,b_n)\}$

$S \overset{d_{\color{blue} i_1}}{\rightarrow} m_1 \overset{d_{\color{blue} i_2}}{\rightarrow} \dots \overset{d_{\color{blue} i_p}}{\rightarrow} m_p$

$m_k = u_{k+1} {\color{red} a_{i_{k+1}}} v_{k+1} \overset{(a_{{\color{blue} i_{k+1}}},b_{{\color{blue} i_{k+1}}})}{\rightarrow} m_{k+1} = u_{k+1} {\color{red} b_{i_{k+1}} } v_{k+1}$ \\
$j_k := \norm{u_k}$ 

{\small \color{DarkBlue}\texttt{type 'e preuveformelle = \\
(('e caractere list)*int*int) list}}

{\small \color{DarkBlue}\texttt{[\dots ;(}}
$m_k$
{\small \color{DarkBlue}\texttt{,}}
$\color{blue} i_k$
{\small \color{DarkBlue}\texttt{,}}
$\color{blue} j_k$
{\small \color{DarkBlue}\texttt{);\dots ]}}
\end{frame}
\begin{frame}\frametitle{Vérification de preuve - Exemple}
$G = (T,N_t,S,D)$ où :
\begin{itemize}
\item $T = \{\underline{ a },\underline{ b },\underline{ c }\}$
\item $N_t = \{\underline{ S },\underline{ B }\}$
\item $D = \{(\underline{ S },\underline{ aBSc })_1,(\underline{ S },\underline{ abc })_2,(\underline{ Ba },\underline{ aB })_3,(\underline{ Bb },\underline{ bb })_4\}$
\end{itemize}

alors \underline{ aabbcc } est dans $\mathcal{L}(G)$ car $\underline{ S } \rightarrow_1 \underline{ aBSc } \rightarrow_2 \underline{ aBabcc } \rightarrow_3 \underline{ aaBbcc } \rightarrow_4 \underline{ aabbcc }$.

En Ocaml :

{\small \color{DarkBlue}\texttt{let unepreuve = [(mot "aBSc",0,0);(mot "aBabcc",1,2);
		 (mot "aaBbcc",2,1);(mot "aabbcc",3,2)]}}
\end{frame}

\begin{frame}\frametitle{Preuve automatique d'appartenance d'un mot}
Chemin de $S$ à $m$ dans $(\Sigma^\star,\rightarrow)$.
\pause

Parcours en largeur ! Successeurs directs $\mathcal{S}(x)$ d'un mot $x$ par $\rightarrow$ ?
\pause

Algorithme de Knuth-Morris-Pratt pour le calcul de $\mathcal{S}_{(a,b)}(x)$ successeurs directs de $x$ par $\overset{(a,b)}{\rightarrow}$.
\end{frame}

\begin{frame}\frametitle{Preuve automatique d'appartenance d'un mot - Le parcours en largeur}
Réduire la complexité d'un tel parcours ?

Eviter les chemins passant par certains mots.

{\small \color{DarkBlue}\texttt{interdit : ('e caractere list) -> bool\\}}

Retourner le chemin pour la preuve d'appartenance du mot.

\end{frame}

\begin{frame}\frametitle{Amélioration pour les grammaires croissantes}

\begin{equation*}\forall (a,b) \in D, \norm{a} \leq \norm{b} \end{equation*}

\begin{equation*}x \overset{*}{\rightarrow} x' \implies \norm{x} \leq \norm{x'}\end{equation*}

{\small \color{DarkBlue}\texttt{let interditcroiss x = \\ Array.length x > Array.length m}}
\end{frame}

\begin{frame}\frametitle{Amélioration dans le cas général : déduction sur le nombre d’occurence de chaque lettre}
\texttt{ interdit } plus sophistiquée dans le cas général.

$l \in \Sigma$, $s_x(l) \supset \norm{\delta(x)}_l$ nombres d'occurences de $l$ dans les successeurs de $x$. 

Ainsi, $x \overset{*}{\rightarrow} m \implies \forall l \in \Sigma, \norm{m}_l \in s_x(l)$
\pause

Contraposée : 
\fbox{$\exists l \in \Sigma / \norm{m}_l \notin s_x(l) \implies \neg (x \overset{*}{\rightarrow} m)$}
\pause

Pour les grammaires croissantes, $s_x(l) \supset [|\norm{x},\infty|[$
\end{frame}

\begin{frame}\frametitle{Calcul de $s_x$ - Description des mots}
Description d'un mot $x$ (par ses nombres d'occurences des lettres).

Ensemble de ces descriptions
$Q$ fini $ \subset ({\mathcal{P}(\mathbb{N})})^{\Sigma}$.

Soit $cat(x) \in Q$ la description de $x$,

$\forall l \in \Sigma, \norm{x}_l \in cat(x)(l)$.
\pause

Exemple : pour $Q = \{\{0\},\mathbb{N}^*\}^\Sigma$,
\begin{align*}
cat(\underline{ abcba }) : & a \mapsto \mathbb{N}^*\text{, }b \mapsto \mathbb{N}^*\text{,}\\
&c \mapsto \mathbb{N}^*\text{, }d \mapsto \{0\}\text{,}\\
&S \mapsto \{0\}
\end{align*}
\pause

Partition de $\Sigma^\star$ par $x \sim y \iff cat(x) = cat(y)$.
\end{frame}

\begin{frame}\frametitle{Calcul de $s_x$ - Le graphe $(Q,A_0)$}
Graphe $(Q,A_0)$ où $A_0 = cat(\rightarrow)$, $cat$ homomorphisme de graphes.

$x \rightarrow x' \implies (cat(x),cat(x')) \in A_0$
\pause

$(q,q') \in A_0$\\ 
$\iff \exists x,x' \in \Sigma^\star / x \rightarrow x' \text{ et } cat(x) = q \text{ et } cat(x') = q'$

$A_0$ dérivations possibles d'une description $q \in Q$ à une autre $q' \in Q$.
\end{frame}
\begin{frame}\frametitle{Calcul de $s_x$}
$x \overset{*}{\rightarrow} m \implies cat(m)$ accessible depuis $cat(x)$ dans tout $(Q,A)$ où $A \supset A_0$.
\pause

La contraposée :
\fbox{\parbox{\linewidth}{
    Si $cat(m)$ n'est pas accessible depuis $cat(x)$ dans un graphe $(Q,A)$ majorant $(Q,A_0)$, 
    alors $\neg (x \overset{*}{\rightarrow} m)$.
}}
\pause

(Ici $s_x(l) = \bigcup_{y / x\overset{*}{\rightarrow} y} cat(y)(l) = \bigcup_{q\text{ accessible depuis } cat(x) \text{ dans A}}q(l)$).

\end{frame}

\begin{frame}\frametitle{Les états $q$ sous la forme $q \in Q = \{\{0\},\mathbb{N}^*\}^{\Sigma}$}

$D = \{(S,\underline{ abc })_0,(\underline{ abc },\underline{ ab })_1,(\underline{ b },\underline{ k })_2,(\underline{ c },\underline{ ak })_3,(\underline{ kak },\underline{ aa })_4,(\underline{ a },\underline{ aaa })_5\}$

\begin{figure}
    \centering
    \subfloat[\centering Le graphe A]{\scalebox{0.7}[0.7]{\tikzfig{grosgraphe}}}
    \qquad
    \subfloat[\centering Ses sommets]{\hspace{-2cm}    \tiny
    \begin{tabular}{|L{0.2cm}||L{0.2cm}|L{0.2cm}|L{0.2cm}|L{0.2cm}|L{0.2cm}|}
    \hline
    q&q(a)&q(b)&q(c)&q(k)&q(S)\\\hline
    $\alpha$&$\mathbb{N}^*$&$\mathbb{N}^*$&$\mathbb{N}^*$&$\{0\}$&$\{0\}$\\\hline
    $\beta$&$\mathbb{N}^*$&$\mathbb{N}^*$&$\{0\}$&$\{0\}$&$\{0\}$\\\hline
    $\gamma$&$\mathbb{N}^*$&$\{0\}$&$\mathbb{N}^*$&$\{0\}$&$\{0\}$\\\hline
    $\delta$&$\mathbb{N}^*$&$\mathbb{N}^*$&$\{0\}$&$\mathbb{N}^*$&$\{0\}$\\\hline
    $\epsilon$&$\mathbb{N}^*$&$\{0\}$&$\{0\}$&$\mathbb{N}^*$&$\{0\}$\\\hline
    $\varphi$&$\mathbb{N}^*$&$\{0\}$&$\{0\}$&$\{0\}$&$\{0\}$\\\hline
    $\psi$&$\mathbb{N}^*$&$\{0\}$&$\mathbb{N}^*$&$\mathbb{N}^*$&$\{0\}$\\\hline
    $\mu$&$\mathbb{N}^*$&$\mathbb{N}^*$&$\mathbb{N}^*$&$\mathbb{N}^*$&$\{0\}$\\\hline
    $\sigma$&$\{0\}$&$\{0\}$&$\{0\}$&$\{0\}$&$\mathbb{N}^*$\\\hline
    \end{tabular}}
\end{figure}

\end{frame}

\begin{frame}\frametitle{Les états $q$ sous la forme $q \in Q = \{\{0\},\mathbb{N}^*\}^{\Sigma}$}

\tikzfig{ptitgraphe}

$\neg$ (\underline{ aaabakab } $\overset{*}{\rightarrow}$ \underline{ akkcckaaakck }) ($\psi$ pas accessible depuis $\delta$)

\end{frame}

\begin{frame}\frametitle{Calcul des arêtes $A$ du graphe, conditions nécessaires}
$\exists x,y / cat(x) = q \text{ et } cat(y) = q \text { et } x \overset{(a,b)}{\rightarrow} y$\\
$\implies$
\begin{itemize}
\item $cat(a) \preceq cat(x)$ où\\ $\forall \alpha, \beta \in Q, (\alpha \preceq \beta) \iff \forall l \in \Sigma, \max \alpha(l) \leq \min \beta(l)$.
\pause

\item $cat(x)-cat(a) \preceq cat(y)$ où\\ $\forall \alpha,\beta \in Q, \alpha-\beta : l \mapsto 
\begin{cases}
\{0\} & \text { si } q'(l) = \mathbb{N} \text{ ou }q(l) = \{0\}\\
\mathbb{N} & \text{sinon}
\end{cases}
$
\pause

\item $\forall l \in \Sigma, cat(b)(l) = \mathbb{N}^* \implies cat(y)(l) = \mathbb{N}^*$. 
\pause

\item $\forall l \in \Sigma, cat(x)(l) = \{0\}$ et $cat(b)(l) = \{0\} \implies cat(y)(l) = \{0\}$.

\end{itemize}

\end{frame}

\begin{frame}\frametitle{Calcul des arêtes $A$ du graphe (3)}
Pour
$A_{(a,b)} \supset \{(q,q') \in A \mid \exists x,x' \in \Sigma^\star / x \overset{(a,b)}{\rightarrow} x' \text{ et } cat(x) = q \text{ et } cat(x') = q'\}$

$A = \bigcup_{(a,b) \in D}  A_{(a,b)}$.

\end{frame}

\begin{frame}\frametitle{Conclusion}
En précalculant la matrice d'accessibilité de A (avec Floyd-Warshall), \texttt{ elimine x } en $\mathcal{O} (|x|)$.
\pause

Mais, seulement 3 appels à \texttt{ inderdit } réussis pour dériver \underline{ S } en \underline{ aaakaaak } de profondeur 5...
\pause

Prétraitement couteux
\pause

Extension probablement possible (mais aussi peu utile ?) pour $Q \subset \{2 \mathbb{N} , 2 \mathbb{N} + 1 \}^\Sigma$

\end{frame}


\end{document}
    