\newcommand{\letitle}{Sur la vérification de preuve et la preuve automatique d'appartenance d'un mot à une grammaire.}
\newcommand{\leauthor}{Ulysse Durand}

\documentclass[12pt]{beamer}
\usepackage{tikzit}

\usepackage[utf8]{inputenc}
\usepackage{amssymb}
\usepackage{amsmath}
\usepackage{xcolor}
\usepackage{enumitem}
\usepackage{bbold}
\usepackage[xcolor,leftbars]{changebar}

\setcounter{secnumdepth}{1}
\newenvironment{myindentpar}
 {\begin{list}{}
         {\setlength{\leftmargin}{1em}}
         \item[]
 }
 { \end{list}}

\definecolor{DarkBlue}{RGB}{0,16,80}
\newcommand{\norm}[1]{\lvert #1 \rvert}
\newsavebox{\mybox}
\newlength{\mydepth}
\newlength{\myheight}
\newenvironment{answer}
{\par\begin{lrbox}{\mybox}\quad\begin{minipage}{\linewidth}\color{black}\setlength{\parskip}{10pt plus 1pt minus 1pt}\vspace*{-.7\baselineskip}}
{\end{minipage}\end{lrbox}
\settodepth{\mydepth}{\usebox{\mybox}}
\settoheight{\myheight}{\usebox{\mybox}}
\addtolength{\myheight}{\mydepth}
\noindent\makebox[0pt]{
  \color{gray}\hspace{-0pt}\rule[-\mydepth]{1pt}{\myheight}}
  \usebox{\mybox}
  }

\usepackage{hyperref}
\setlength{\parskip}{0.15cm}
\hypersetup{
    colorlinks=truem,
    linkcolor=black,
    filecolor=red,
    urlcolor=blue
}

\urlstyle{same}

\everymath{\displaystyle}
\title{\letitle}
\author{\leauthor}
\date{}
\begin{document}
\begin{frame}
\titlepage
\end{frame}
    \begin{frame}\frametitle{Les grammaires formelles}

$G = (T,N_t,S,D)$ où :

\begin{itemize}
\item $T$ est l'alphabet des terminaux
\item $N_t$ est l'alphabet des non terminaux
\item $S \in N_t$ est l'axiome

Notons $\Sigma := N_t \cup T$
\item $D \subset \mathcal{P}((\Sigma ^ \star )^2)$, est l'ensemble des règles de dérivation.
\end{itemize}
\end{frame}

\begin{frame}\frametitle{Definitions}
$\forall x,x' \in \Sigma^\star, x \overset{(a,b)}{\rightarrow} x' \iff \exists u,v \in \Sigma^\star / x = uav \text{ et } x' = ubv$

$\rightarrow := \bigcup_{d \in D} \overset{d}{\rightarrow}$ et on note $\overset{*}{\rightarrow}$ la cloture transitive et réflexive de $\rightarrow$

$\delta(x) := \{y \in \Sigma^\star / x \overset{*}{\rightarrow} y\}$

$\norm{x}_l := \norm{\{i \in \mathbb{N} \mid x_i = l \}}$ est le nombre d'occurences de la lettre $l$ dans x.

Alors le langage de la grammaire formelle  G est le suivant :
\begin{equation*}\mathcal{L} (G) := \delta(S) \cap T^\star\end{equation*}
\end{frame}

\end{document}
    